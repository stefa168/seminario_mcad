\documentclass{beamer}
\usetheme{Madrid}

\usepackage{adjustbox}
\usepackage[absolute, overlay]{textpos}
\usepackage[utf8]{inputenc}


%Information to be included in the title page:
\title{Impossibilita del consenso nel modello a rete asincrono, una soluzione randomizzata}
\subtitle{Seminario 8}
\author{Daniele Liberatore, Matteo Palazzo, \\ Stefano Porta, Alberto Mulone}
\institute{Università degli Studi di Torino}
\date{20 Gennaio 2021}

\begin{document}

\frame{\titlepage}

\begin{frame}{Overview}
    \tableofcontents
\end{frame}

% Capitolo 17

\begin{frame}{Modello a rete asincrona}
    
\end{frame}

\section{Da modello a rete al modello a memoria condivisa}

\begin{frame}{Da modello a rete al modello a memoria condivisa}
    \begin{block}{Perchè convertire un modello a rete in uno a memoria condivisa?}
        \begin{itemize}
            \item I sistemi a memoria condivisa sono di più facile comprensione e offrono una maggiore espressività.
            \item Esistono numerosi algoritmi già sviluppati per i modelli a memoria condivisa.
            \item Il modello a rete simula ed eredita le caratteristiche proprie di un sistema a memoria condivisa.
        \end{itemize}
    \end{block}
\end{frame}

\subsection{Il problema del consenso}

\begin{frame}{Impossibilità del consenso nel modello a rete}

    Il modello a rete eredita dal modello a memoria condivisa l'impossibilita del consenso.

    \begin{block}{Teorema}
        Dato un qualsiasi sistema asincrono a rete composto da un numero $n \geqslant 2$ di processi, non esiste un algoritmo che risolva del consenso e garantisca la \textit{1-failure termination}.
    \end{block}
\end{frame}

\begin{frame}{Concetti chiave}
    \begin{itemize}
        \item Fault-tollerance %% Spiegare un evento di stop
        \item I-simulazione
    \end{itemize}    
\end{frame}

\begin{frame}{Fault-tollerance}
    
\end{frame}

\begin{frame}{I-simulazione}
    
\end{frame}

\begin{frame}{SimpSRSim}
    
\end{frame}

\begin{frame}{SimBCast}

\end{frame}

% SimpSRSim
% SimpBCastSim

\subsection{Equivalenza del modello Send/Receive e il modello Broadcast}

\begin{frame}{Equivalenza tra modello Send/Receive e Broadcast}
    \begin{block}{Teorema}
        Dato un generico sistema asincrono a rete broadcast A esiste un sistema asincrono send / receive B che è una I-simulazione di A.
    \end{block}
    
    \vspace{0.5cm}
    
    Data questa equivalenza da ora in poi supporemo di lavorare in sistemi asincroni a rete broadcast, in quanto permettono la stesura di algoritmi più intuitivi e più facilmente dimostrabili.
\end{frame}


\begin{frame}{Impossibilità del consenso nel modello a rete}
    \begin{block}{Dimostrazione}
    \begin{itemize}
        \item Supponiamo per \textbf{assurdo} che esista un algoritmo \texttt{A} in grado di risolvere il problema del consenso in un sistema asincrono a rete broadcast e che garantisca la 1-failure termination. 
        \item Per quanto dimostrato è possibile ottenere un algoritmo B per un sistema a modello a memoria condivisa che è un n-simulazione di \texttt{A}. 
        \item Pertanto \texttt{B} risolverebbe il problema del consenso garantendo la 1-failure termination. Questo però contraddice il teorema dell'impossibilità del consenso nel modello a memoria condivisa. 
    \end{itemize}
    \end{block}
\end{frame}

% Capitolo 21

\section{Una soluzione randomizzata}

\begin{frame}{Una soluzione randomizzata}
    Poichè problema del consenso è di vitale importanze in diversi settori, come ad esempio la gestione di transazioni in database distribuiti; è stato necessario sviluppare degli espedienti:
    \vspace{0.2cm}
    \begin{itemize}
        \item Indebolimento dei requisiti di correttezza
        \item Rafforzamento del modello attraverso: 
        \vspace{0.2cm}
        \begin{itemize}
            \item \textbf{Utilizzo della randomizzazione}
            \item Failure detection
            \item Consenso su un insieme di valori
            \item Consenso parziale
        \end{itemize}
    \end{itemize}
\end{frame}

\begin{frame}{Requisiti di correttezza}

\end{frame}

\subsection{Algoritmo di BenOr}


\end{document}